\chapter{Discussion and Analysis of Skills Acquired}

\section{Introduction}
This chapter analyzes the skills gained during the SIWES training. It covers UI/UX design, LaTeX typesetting, and professional competencies, reflecting on how these skills were developed through the projects described in Chapter 2.

\section{Analysis of Skills and Experience Gained}

\subsection{UI/UX Design Skills}

\subsubsection{Progression}
Skill development progressed significantly from basic training to professional-level work. The journey began with foundational training focused on basic shapes and the pen tool, where consistent practice helped build confidence in creating vector graphics. This led to an assessment phase where the burial invitation project tested the ability to select appropriate styles for sensitive contexts, requiring a careful balance of aesthetics and tone.

As confidence grew, the work transitioned to real client projects. The graduation invitation introduced the challenges of direct client interaction and feedback management, which are critical for professional design work. Finally, the Litas Travels and Attendance System projects demonstrated the ability to design complete, responsive systems from scratch, marking a shift to handling complex, multi-page interfaces.

\subsubsection{Technical Skills}
The internship provided the opportunity to master several key technical skills. A major focus was learning to use reusable components, such as buttons and cards, which ensured design consistency and improved efficiency across large projects. Mastering auto-layout was another critical step, enabling the creation of responsive designs that adapt seamlessly to different screen sizes.

Prototyping skills were also developed by creating interactive flows with dropdowns and dialogs to demonstrate user journeys effectively. Additionally, applying visual principles like alignment, spacing, and color theory helped create professional-looking interfaces that are both functional and visually appealing.

\subsubsection{Professional Growth}
Beyond technical skills, there was significant professional growth. Designing for specific emotional contexts, such as the somber tone required for burial cards, improved context awareness. Client communication skills also sharpened, particularly in understanding needs and managing expectations during feedback cycles. Furthermore, planning complex projects with wireframes before starting the actual design became a standard practice, ensuring a more organized and efficient workflow.

\subsection{LaTeX Typesetting Skills}

\subsubsection{Document Variety}
Handling diverse document types required adapting to different formatting approaches. Academic work involved learning specific humanities citation styles and managing complex footnotes. Legal documents demanded an extreme attention to detail with zero tolerance for errors, ensuring that every clause and line number was perfectly placed.

Mathematical documents presented a different challenge, requiring the mastery of complex syntax to typeset exams and equations accurately. Religious documents, such as church programs, required balancing dense information with a readable layout, ensuring the final output was both comprehensive and legible.

\subsubsection{Technical Depth}
Technical proficiency in LaTeX grew through the extensive use of various packages. Proficiency was gained in using \texttt{amsmath} for mathematics, \texttt{biblatex} for citations, \texttt{geometry} for page layout, and \texttt{multicol} for column management.

There was also a strong focus on mathematical notation, building the confidence to typeset complex equations, matrices, and integrals. Structurally, skills were developed in managing tables of contents, customizing page numbering styles, and generating automated bibliographies, which are essential for long-form professional documents.

\subsubsection{Adaptability}
A key skill developed during this period was adaptability. The ability to quickly learn new formats—such as switching to humanities citations for a specific project—and mastering specialized syntax for mathematics demonstrated a capacity to handle a wide range of typesetting requirements effectively.

\subsection{Professional Competencies}

\subsubsection{Productivity and Precision}
Workplace demands highlighted the importance of both speed and accuracy. Reaching a typing speed of 90 WPM demonstrated the ability to work fast under pressure, which was essential for meeting urgent deadlines. However, work on legal documents emphasized that speed cannot come at the expense of precision; absolute accuracy and careful proofreading were paramount. Balancing this need for speed with the requirement for high-quality output became a core professional competency.

\subsubsection{Equipment and Workflow}
Practical experience with office equipment included learning to troubleshoot common issues like printer jams and adjusting settings for different paper materials. Efficient workflows were also developed to handle batch printing tasks and manage customer queues during peak hours, ensuring smooth operations even when the office was busy.

\section{Tools and Equipment Used}

\subsection{Figma Analysis}
Figma served as the primary design tool throughout the internship.

\subsubsection{Strengths}
The tool's real-time collaboration features are excellent for teamwork, allowing multiple designers to work simultaneously, although this internship primarily utilized these features for supervisor feedback. For larger projects like the Attendance System, component libraries proved essential. They ensured consistency across all screens and significantly improved efficiency by allowing the reuse of elements like buttons and navigation bars.

Figma's prototyping capabilities also allowed for the creation of interactive demos rather than static images, which was crucial for visualizing user flows. Additionally, the auto-layout feature was indispensable for creating responsive designs that remain organized and scalable across different device sizes.

\subsubsection{Progression}
Learning in Figma followed a clear trajectory from basic to advanced skills. Initially, the focus was on understanding basic shapes, text manipulation, and mastering the pen tool for vector graphics. As proficiency grew, the focus shifted to intermediate skills like using auto-layouts and managing multi-page designs. By the end of the internship, advanced features such as variables, complex components, and high-fidelity interactive prototypes were being utilized to build sophisticated systems.

\subsubsection{Relevance}
As an industry standard, proficiency in Figma is a strong asset. The skills developed—ranging from basic UI construction to complex system management—are directly transferable to any professional design role.

\subsection{LaTeX Evaluation}
LaTeX was established as the primary tool for creating professional, high-quality documents.

\subsubsection{Strengths}
Its superior mathematical support makes it the standard for technical documents, offering a professional look that word processors struggle to match. Consistency is another major advantage; LaTeX's automatic formatting ensures that headers, lists, and fonts remain uniform throughout the document. Automation features, such as auto-generated tables of contents, bibliographies, and page numbering, save significant time and reduce the likelihood of human error. Furthermore, its standardization makes it essential for academic work where strict adherence to formatting guidelines is required.

\subsubsection{Word vs. LaTeX}
While Microsoft Word is generally easier for simple, day-to-day documents, LaTeX excels in specific areas. It is superior for academic papers due to its robust handling of citations and structural elements. For legal documents, its precision and compatibility with version control systems make it the better choice. It is also the only viable option for math-heavy documents where complex equations need to be rendered clearly.

\subsubsection{Applications}
Developing skills in LaTeX opens up opportunities in several fields, including academic publishing, technical writing, and legal documentation. The ability to manage complex, structured documents is a valuable specialized skill.

\subsection{Hardware Assessment}

\subsubsection{HP LaserJet Printers}
The HP LaserJet printers demonstrated exceptional reliability for high-volume work, handling large batch printing tasks without issues. Through daily operation, valuable troubleshooting skills were acquired, including clearing paper jams and managing toner levels to ensure continuous operation. The output quality was consistently high, producing sharp, legible text suitable for professional academic and legal use.

\subsubsection{Laminating Machine}
Operating the laminating machine required careful attention to detail to achieve good results. precise quality control was necessary to ensure that documents were free of bubbles or wrinkles, which could ruin important certificates. Additionally, managing heat settings was critical, as different paper thicknesses required specific temperature adjustments to prevent warping.

\subsubsection{Summary}
Operating this office equipment provided practical hands-on experience. The skills gained in troubleshooting and maintenance are useful in any office setting, demonstrating capability in managing physical workflows.

\section{Challenges Encountered and Solutions}

\subsection{Design Challenges}

\subsubsection{Mastering Curves}
The pen tool presented an initial challenge, as it was difficult to control for precise vector work. To overcome this, repeated practice sessions were undertaken, focusing on creating complex shapes. This dedication resulted in gaining precise control over vector paths and a deeper understanding of digital geometry.

\subsubsection{Color Matching}
Matching print colors to screen colors without digital hex codes proved difficult. A trial-and-error approach involving printing multiple variations and comparing them under different lighting conditions solved this. This experience provided valuable insight into the complexities of color management between digital and physical mediums.

\subsubsection{Designing from Scratch}
Creating the Litas Travels UI without any reference materials was daunting. To address this, extensive research into existing travel websites was conducted to identify common patterns and user expectations. This taught the valuable lesson that research should drive design decisions, transforming a blank canvas into a structured project.

\subsubsection{Finding Assets}
Finding respectful and appropriate icons for burial cards was challenging due to the limited options in standard libraries. The solution involved careful searching and curation of assets from specialized sources. This highlighted the importance of not relying solely on default libraries and the need to curate resources to match the specific tone of a project.

\subsection{Typesetting Challenges}

\subsubsection{New Citation Styles}
Learning the humanities footnote style quickly for a specific project was a significant hurdle. By consulting documentation and style guides, the necessary formatting rules were mastered rapidly. This developed confidence in the ability to learn new formats on the fly and adapt to client requirements.

\subsubsection{Legal Precision}
The requirement for zero tolerance for errors in legal texts was demanding. A system of systematic proofreading and rigorous change tracking was implemented to ensure accuracy. This process cultivated a habit of extreme attention to detail which is critical for high-stakes professional work.

\subsubsection{Math Syntax}
Memorizing and using complex LaTeX math commands initially slowed down the workflow. Through consistent practice and experimentation with different equations, proficiency was achieved. This resulted in the ability to confidently handle technical typesetting tasks involving complex mathematical notation.

\subsubsection{Guidelines Compliance}
Meeting strict university formatting rules for project reports required meticulous care. By carefully checking the document against the requirements line by line, the challenge was met, resulting in success in first-time submissions without the need for major revisions.

\subsection{Operational Challenges}

\subsubsection{High Volume Printing}
Managing large batches of student clearance slips simultaneously was overwhelming. To manage this, an organized batch printing workflow was established, prioritizing jobs and managing paper flow. This developed the ability to handle high-pressure volume efficiently without compromising on organization.

\subsubsection{Equipment Maintenance}
Preventing printer overheating and jams during long print runs was a constant concern. Proactive monitoring and ensuring correct settings for paper types prevented most issues. This experience built practical hardware maintenance skills that are essential for keeping an office running smoothly.

\subsubsection{Queue Management}
Balancing multiple urgent customer requests during peak times was stressful. Clear communication about wait times and prioritizing tasks based on urgency helped manage customer expectations. This significantly improved customer service skills and the ability to remain calm under pressure.

\subsubsection{Speed vs. Accuracy}
Typing fast (90 WPM) without making errors was a difficult balance to strike. Maintaining focus and taking brief breaks to stay fresh allowed for sustained high speed without a drop in quality. This proved that speed and accuracy can coexist with the right mental approach.

\section{Conclusion}
The SIWES training provided comprehensive professional development that extended far beyond technical skills. Proficiency was gained in industry-standard tools like Figma and LaTeX, while design, typesetting, and equipment operation skills were refined through practical application. More importantly, the experience taught professionalism—how to handle pressure, manage client expectations, and solve problems systematically. It was not just about learning software; it was about learning to work effectively as a professional in a demanding environment.

