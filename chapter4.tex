
\chapter{Conclusion, Limitations and Recommendations}

\section{Conclusion}

The SIWES training at Litashub Cyber Cafe gave broad experience in UI/UX design, document preparation, customer service, and office operations. During the training, the intern moved from learning to working independently on client projects, showing significant growth in several skills.

\subsection{Summary of Training Experience}
The internship covered three distinct but interconnected areas: UI/UX design, document typesetting, and general office operations. In design, the progression was clear, moving from basic tool familiarity in Figma to executing complex projects like the Litas Travels landing page and the Attendance System mobile app. This evolution from simple shape manipulation to full-scale system design demonstrated a significant growth in technical capability.

Simultaneously, the typesetting work with LaTeX highlighted the importance of precision and structure. Handling a wide variety of documents—from academic papers with strict citation styles to legal documents requiring zero-tolerance accuracy—provided a deep understanding of professional documentation standards.

Complementing these technical skills were practical experiences in customer service and office operations. Daily interactions with diverse customers sharpened communication skills and patience, while operating high-volume printers and laminators built practical troubleshooting abilities. The achievement of a 90 WPM typing speed stand as a testament to the productivity gains made during the program.

\subsection{Relevance to Academic and Professional Growth}
The training served as a vital bridge between classroom theory and real-world application. Concepts from software engineering and user experience design were directly applied to client projects, validating academic learning in a practical setting. For instance, the Litas Travels project required genuine design thinking, while the preparation of legal documents reinforced the precision and attention to detail emphasized in engineering disciplines. Furthermore, specialized tools like Figma and LaTeX provided a competitive edge, as proficiency in these industry standards is highly valued in design and technical roles. Beyond technical skills, the training fostered professional maturity; learning to manage client expectations, handle feedback constructively, and work effectively under pressure provided a solid foundation for a successful career.

\subsection{Overall Success of Training}
The training program successfully met its core objectives by effectively connecting theoretical knowledge with practical experience. It facilitated valid skill acquisition in key areas such as interference design, professional typesetting, and office management, while also offering valuable insights into business operations. The successful completion of major projects, backed by positive client feedback, serves as tangible proof of the high standards achieved. Ultimately, the internship has prepared the intern for a professional career, equipping them with a robust mix of technical skills, problem-solving abilities, and professional confidence.

\section{Limitations}
While the internship was valuable, there were some limitations:

\subsection{Limited Exposure to Advanced Design Projects}
While the design projects undertaken were of high quality, they were primarily focused on standard deliverables such as invitation cards and landing pages. Extremely complex tasks, such as large-scale web platforms involving intricate data visualization or specialized areas like game UI and VR, were not explored. Although the attendance system was a complex mobile app, it remained within the bounds of standard application design. Additionally, the small team size at Litashub meant that there was no opportunity for team-based design workflows involving collaboration with other designers or developers, which is a common aspect of larger corporate environments.

\subsection{Project Focus and Scope}
Certain advanced features of the tools used were not fully utilized simply because the projects did not require them. For instance, advanced LaTeX packages for scientific plotting or complex Figma prototyping features were not explored in depth. While the skills acquired are solid for core professional tasks, these advanced capabilities remain areas for future self-directed learning.

\subsection{Hardware and Software Limitations}
The equipment at Litashub, while reliable, was somewhat dated. The printers lacked modern capabilities such as color printing and booklet finishing, meaning experience with these features was missed. Similarly, access to paid design resources was limited, often necessitating manual searching for assets where a subscription library would have been more efficient. Furthermore, the experience was confined to a small business environment; working in a larger enterprise would have provided different insights into corporate structure and processes.

\subsection{Exposure to Specialized Design Disciplines}
The internship's design focus was heavily centered on UI/UX. Other important disciplines such as user research, formal accessibility compliance, motion design, and print-specific design methodologies were not deeply explored. These represent key areas for future study and professional development.

\section{Recommendations}

\subsection{For Future Interns}
Future interns would benefit significantly from completing formal training on key tools before strictly commencing client work. Dedicating the first week to intensive tutorials would likely be more efficient than the "learning while working" approach. Additionally, maintaining a comprehensive portfolio from day one is highly recommended, as it greatly assists with future job applications and self-assessment. Regular, structured feedback sessions with supervisors would also help accelerate the learning curve.

\subsection{For Litashub Management}
Creating a structured internship plan with clear weekly milestones would greatly improve the learning experience. Additionally, documenting standard operating procedures for common tasks would help new interns onboard more quickly and independently. regarding equipment, considering an upgrade to modern printers with color capabilities would allow the business to offer a wider range of services and provide interns with more relevant experience.

\subsection{For Academic Institutions}
Academic institutions should aim to include more realistic, project-based work in their curricula to better prepare students for industry demands. Specifically, teaching professional documentation tools like LaTeX as part of the standard coursework would give students a significant head start in their professional lives.

\section{Final Reflections}
The SIWES training at Litashub Cyber Cafe effectively connected academic knowledge with professional work. The experience in design, typesetting, and customer service provided a complete look at a professional environment.

The training built essential skills like tool mastery, client communication, and problem-solving. Successes with the Litas Travels landing page and attendance system show the high quality of work achieved. Positive client feedback confirms this value.

The limitations, such as project scope and time, are normal for an internship and point to areas for future growth.

Most importantly, the internship showed that professional design work matches the intern's interests and skills. Handling real-world problems and client needs built both technical skill and confidence.

The intern finishes the program with a strong foundation for a career in design or technology. The drive to learn and improve suggests a promising professional future.


