
\chapter{Conclusion, Limitations and Recommendations}

\section{Conclusion}

The Students Industrial Work Experience Scheme (SIWES) training at Litashub Cyber Cafe provided comprehensive exposure to professional practice in UI/UX design, document preparation, customer service, and office operations. Over the training period, the intern transitioned from observation and guided practice to independent responsibility for client-facing work, demonstrating significant professional growth across multiple competency areas.

\subsection{Summary of Training Experience}

The internship encompassed three major professional domains. UI/UX design work in Figma progressed systematically from foundational training with basic shapes and vector tools, through skill assessment with the burial invitation card project, to independent client work with the graduation card design, and culminating in complex professional projects including the Litas Travels landing page and a complete nine-screen mobile application interface for an attendance system. This progression demonstrated clear mastery development from beginner to intermediate-to-advanced proficiency.

LaTeX typesetting responsibilities revealed the tool's versatility across diverse professional contexts. Work ranged from academic document preparation (Ph.D assignments with humanities citation formats) to high-stakes legal document formatting requiring absolute precision and zero-tolerance revision cycles, educational content preparation (mathematics examination papers with complex notation), and specialized formatting tasks (church program bulletins). The breadth of document types handled—academic, legal, mathematical, religious, and institutional—required developing specialized competencies in each context.

Customer service and operational responsibilities built practical professional skills. Direct interaction with customers across multiple project types developed communication abilities, patience under pressure, and responsiveness to diverse needs. Equipment operation including high-volume printing, specialized materials handling (certificate cardstock and lamination), and technical troubleshooting built hands-on competency in production environments. Queue management during peak demand periods and the achievement of the 90 WPM typing milestone demonstrated development of professional productivity standards.

\subsection{Relevance to Academic and Professional Growth}

The training directly bridged theoretical knowledge from the Computer Science and Information Technology curriculum with real-world professional practice. Academic studies in user experience principles, software design methodologies, and documentation standards gained practical context and applicability through client work. The experience of designing from scratch (Litas Travels project) required applying design thinking methodologies learned in academic courses. The precision requirements of legal document preparation paralleled quality assurance principles taught in software engineering courses. The systematic approach to managing complex projects (the attendance system interface with advance planning and component systems) reflected professional project management practices.

The training exposed the intern to modern industry-standard tools (Figma for design, LaTeX for technical documentation) that represent valuable professional credentials. Achieving intermediate-to-advanced proficiency in these tools during the internship period creates competitive advantage in the job market for design-focused or technical writing roles.

Beyond technical skills, the training developed professional maturity. Managing client expectations, handling feedback and revision cycles, working under pressure with tight deadlines, maintaining quality standards despite time constraints, and developing systematic approaches to solving unfamiliar problems—these represent the professional competencies that distinguish entry-level employees from those who advance rapidly in their careers.

\subsection{Overall Success of Training}

The training achieved its primary objectives. Bridging theory and practice was accomplished through direct client work in both design and documentation domains. Skill acquisition across multiple tool sets (Figma, LaTeX, office equipment, customer service) was substantial and demonstrated through completed projects and client satisfaction. Industry exposure included understanding business operations, customer dynamics, competitive pressures, and the real-world constraints that shape professional work differently than academic projects.

The projects completed during training—particularly the Litas Travels landing page and the nine-screen attendance system interface—represent portfolio-quality work demonstrating professional-level design capability. The positive feedback from clients (the graduation card customer's satisfaction and tip, the church organization's satisfaction with certificates, the student's successful project report submission) indicates that work quality met or exceeded professional standards.

The internship successfully prepared the intern for professional practice. The combination of technical tool mastery, diverse project experience, client interaction skills, and professional problem-solving competencies creates a foundation for meaningful contribution to professional design or documentation teams.

\section{Limitations}

While the internship provided valuable experience, certain limitations shaped the training experience and should be acknowledged:

\subsection{Limited Exposure to Advanced Design Projects}

The design projects completed, while comprehensive in scope and high quality in execution, primarily focused on relatively straightforward design domains: cards (burial, graduation), landing pages, and mobile application interfaces. More complex design challenges—extensive web platform design with hundreds of screens, complex data visualization interfaces, or design systems for large-scale applications—were not undertaken. 

Additionally, exposure to specialized design domains (game UI design, virtual reality interfaces, accessibility-focused design for users with disabilities, design for non-English language contexts) remained limited. The attendance system project represented the most complex design undertaking but focused on a relatively straightforward mobile application.

The team-based design experience—working in larger design teams, collaborating with developers on design implementation, or participating in design critiques with other designers—was not available in the small-team Litashub environment. Professional growth in these dimensions would require additional experience.

\subsection{Project Focus and Scope}

While the fourteen-week internship duration provided substantial time for skill development, the nature of project-driven learning meant that certain tool features were not explored. LaTeX's complete feature set remains incompletely explored; advanced packages for specialized typesetting and intricate layout control were not undertaken because the project portfolio did not require them. Similarly, Figma's advanced prototyping features (complex animation specifications, conditional properties, or advanced plugin development) were not extensively explored, as the projects undertaken did not demand these capabilities.

The proficiency achieved is intermediate-to-advanced in core design and typesetting functionality used in actual projects. Specialized advanced capabilities in both tools remain unexplored, representing natural extensions for future professional development.

\subsection{Hardware and Software Limitations}

Litashub's LaserJet printer, while reliable, represents older technology. Modern printing equipment includes features for color printing, advanced finishing (booklet creation, stapling), and networked printing that the equipment at Litashub lacked. Experience with contemporary printing technology would expand production capability understanding.

Limited access to advanced design resources or specialized libraries constrained design possibilities in some projects. For example, the burial invitation project required careful searching for appropriate SVG assets; a subscription to comprehensive design asset libraries could have streamlined this process.

The internship provided exposure primarily to small-scale business operations at Litashub. Experience with larger organizational structures, more complex administrative systems, multi-team collaboration, and enterprise-scale tool implementations would provide additional professional context but was not available in the Litashub environment.

\subsection{Exposure to Specialized Design Disciplines}

While UI/UX design was the primary focus, exposure to complementary disciplines remained limited. User research methodologies (user interviews, usability testing, analytics interpretation) were not formally practiced. Design accessibility standards and inclusive design principles, while mentioned in coursework, were not extensively applied to actual projects. Motion design and animation were not part of the design work. Graphic design for print media, while represented by the certificate project, was not a sustained focus.

These specialized areas represent natural extensions of the UI/UX foundation but require additional experience and dedicated study.

\section{Recommendations}

\subsection{For Future Interns}

Future interns should undertake formal tool training before beginning client work. Dedicating the first week to comprehensive Figma and LaTeX tutorials would build feature understanding more efficiently than learning incidentally through projects. Additionally, maintaining a portfolio of best work from day one supports professional development and prepares job market readiness.

Regular feedback sessions with supervisors (weekly or bi-weekly) reviewing completed work and discussing challenges would accelerate skill development more effectively than informal feedback.

\subsection{For Litashub Management}

Establishing a structured internship curriculum with clear skill development milestones and intentionally sequenced project difficulty would improve intern learning compared to ad-hoc project assignment. Additionally, documenting standard operating procedures for common tasks (design revisions, printing procedures, troubleshooting) creates institutional knowledge while supporting intern onboarding.

For equipment, consider upgrading to modern multifunction printers with color and finishing capabilities when replacement is needed, as this would expand service offerings and improve production efficiency.

\subsection{For Academic Institutions}

Academic institutions should integrate real or realistic client projects into design and documentation courses, bridging the gap between purely theoretical education and professional practice. Additionally, expanding curriculum coverage of professional documentation tools (LaTeX, markdown, documentation systems) would better prepare students for industry demands.

\section{Final Reflections}

The SIWES training at Litashub Cyber Cafe provided valuable experience bridging academic knowledge with professional practice. The combination of design work in Figma, diverse typesetting projects in LaTeX, customer service responsibilities, and equipment operation created comprehensive exposure to service-oriented professional environments.

The training successfully developed competencies essential for professional practice: technical tool mastery, client communication, quality standards maintenance, problem-solving under pressure, and systematic approaches to unfamiliar challenges. The completed projects—particularly the Litas Travels landing page and attendance system interface—demonstrate professional-quality work capability. The positive feedback from clients and successful project completions validate the professional value gained.

The limitations identified (restricted project scope, time constraints, organizational scale limitations) are natural to any finite internship period and do not diminish the significant learning achieved. Rather, they identify areas for future professional development and additional experience.

Most significantly, the internship confirmed that professional practice in design and documentation aligns with academic interests and capability. The experiences of managing client expectations, iterating designs based on feedback, handling precision-critical work, and solving real-world problems developed both technical competency and professional confidence.

The intern approaches the conclusion of formal SIWES training with strong foundation for continued professional development. The skills, experience, and confidence gained position the graduate well for professional roles in design, technical documentation, or related technical fields. The commitment to continuous learning, demonstrated through investment in mathematical notation practice and deliberate skill development, suggests a professional trajectory of ongoing growth and increasing responsibility.


