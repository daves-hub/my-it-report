\chapter{Detailed Intern's Role, Responsibilities and Daily Activities}

\section{Introduction}
This chapter provides a detailed description of the roles, responsibilities, and daily activities carried out during the Students Industrial Work Experience Scheme (SIWES) at Litashub Cyber Cafe. The focus is on the practical tasks performed, the tools used, and the nature of work handled throughout the training period. The presentation is descriptive, highlighting UI/UX design, LaTeX typesetting, customer interaction, and the operation of office equipment. Additionally, this chapter delves into specific projects undertaken, challenges encountered, skills acquired, and reflections on the overall experience, providing a comprehensive overview of the internship.

\section{UI/UX Design Activities}
During the industrial training period, the intern was actively involved in UI/UX design activities using Figma. Prior knowledge of UI/UX principles provided a foundation that was strengthened through continuous practice and exposure to real-world design scenarios. Figma's collaborative features allowed for seamless iteration and feedback integration, which was crucial in refining designs based on client input.

Using Figma enabled the creation of structured layouts, organized interface elements, and maintained visual consistency across designs. Activities included simple mobile application screens, website layout designs, and visual mock-ups that demonstrated interface structure and user flow. For instance, one project involved designing a mobile app prototype for a local e-commerce platform, incorporating user-friendly navigation menus, product display grids, and checkout flows to enhance user engagement and conversion rates.

Flyer and poster layouts were also produced for events such as graduation ceremonies and burial announcements. A graduation invitation design was completed and delivered to a client, featuring elegant typography, thematic color schemes, and personalized elements like QR codes for event details. Other designs supported practice and skill improvement, such as creating promotional banners for social media campaigns and redesigning existing website wireframes to improve accessibility.

Throughout these activities, key UI/UX principles such as alignment, spacing, consistency, typography, and thoughtful color usage were applied. Repeated application reinforced best practices and improved overall design quality. Emphasis was placed on clarity and visual appeal to help customers easily understand and approve design outputs. Moreover, user testing simulations were conducted on select designs, gathering informal feedback from colleagues to iterate on usability issues like button placement and information hierarchy.

To further expand on projects, another notable undertaking was the development of a user interface for an internal cafe management tool. This involved creating dashboards for tracking customer orders, inventory management, and service scheduling. The design process included wireframing initial concepts, prototyping interactive elements, and applying responsive design principles to ensure compatibility across desktop and mobile devices. Challenges such as balancing aesthetic appeal with functional simplicity were addressed through multiple revisions, ultimately resulting in a more intuitive system that could potentially streamline cafe operations.

Skills gained from these activities included advanced proficiency in Figma's auto-layout features, understanding of accessibility standards (e.g., WCAG guidelines), and the ability to translate vague client requirements into polished designs. These experiences not only honed technical skills but also emphasized the importance of empathy in user-centered design.

\section{LaTeX Typesetting Activities}
LaTeX typesetting constituted a significant part of the responsibilities at Litashub Cyber Cafe. The work involved preparing and formatting academic and professional documents using LaTeX, a system widely used for producing high-quality structured documents. LaTeX's robustness in handling complex mathematical expressions and bibliographic references made it an ideal choice for the cafe's clientele, primarily consisting of students and academics.

Documents handled included student assignments, examination materials, project reports, and academic notes. The intern also contributed to an open-source mathematical document, gaining exposure to collaborative document development and structured content editing. Both editing existing LaTeX files and creating new documents from scratch were performed. For example, a key project was typesetting a comprehensive undergraduate thesis on computer science topics, which required integrating code listings, algorithms, and data visualizations using packages like listings and tikz.

Common LaTeX features used included sectioning and subsectioning, page numbering, table of contents generation, referencing, insertion of tables and figures, and formatting of mathematical equations. Additional packages such as amsmath for advanced equations, hyperref for clickable links, and biblatex for bibliography management were frequently employed to enhance document functionality.

Litashub adopted LaTeX for academic documents because it produced clean, professional, and standardized outputs. The system was especially useful for project reports submitted by university students, ensuring consistency with institutional requirements. Compared to conventional word processors, LaTeX offered better control over structure, automatic referencing, and time efficiency on large documents. These capabilities helped the organization deliver high-quality outputs that distinguished it from other service providers within the shopping complex.

Another project involved creating a template for SIWES reports, which was customized for multiple students. This template included predefined styles for abstracts, chapters, appendices, and glossaries, incorporating best practices for layout and formatting. Challenges like troubleshooting compilation errors due to mismatched packages or syntax issues were resolved through systematic debugging and consultation with online resources, fostering problem-solving skills.

Furthermore, the intern assisted in converting Microsoft Word documents to LaTeX format, which often required manual adjustments for tables, figures, and footnotes to maintain fidelity. This process highlighted LaTeX's superiority in version control and reproducibility, especially in collaborative environments. Overall, these activities deepened understanding of document automation and prepared the intern for advanced typesetting in academic and professional settings.

\section{Customer Interaction and Service}
Customer interaction formed a major aspect of daily activities at Litashub Cyber Cafe. The intern interacted directly with customers, receiving requests, clarifying requirements, and providing assistance related to typing, printing, design, and document preparation services. Effective communication was key to building trust and ensuring repeat business.

Typical requests included document typing, printing, corrections, redesigns, and minor formatting adjustments. Some customers had clear specifications, while others needed guidance on suitable layouts or presentation styles. In such cases, options were communicated and design or formatting choices explained to ensure satisfaction. For instance, when a client requested a resume redesign, the intern suggested modern templates with infographics to highlight skills, leading to a more impactful document.

Customers frequently requested changes after reviewing completed outputs, either due to revised preferences or initial oversight. This required attentiveness, patience, and flexibility while handling multiple tasks simultaneously. At times, several customers were attended to concurrently, improving multitasking ability, time management, and interpersonal communication skills. Guidance from senior staff further enhanced professional customer service and effective problem handling.

Additionally, cultural sensitivity played a role in interactions, as customers from diverse backgrounds visited the cafe. Learning to adapt communication styles accordingly improved service quality. Reflections on these experiences underscore the value of soft skills in technical roles, preparing the intern for future professional environments.

\section{Operation of Office Machines}
Beyond design and typesetting tasks, the intern operated office machines used at Litashub Cyber Cafe. Standard HP LaserJet printers and photocopy machines supported daily operations, ensuring smooth workflow.

Routine tasks included printing documents, reloading paper trays, adjusting printer settings for different paper sizes and orientations, and clearing minor paper jams. Photocopying was performed as required by customers. The intern also operated the laminating machine for document finishing and protection. These activities built familiarity with basic office equipment and routine maintenance procedures.

Expanding on projects, the intern managed a bulk printing job for conference materials, which involved collating multiple copies, binding them using spiral machines, and ensuring quality control. This project required coordination with suppliers for paper stock and troubleshooting equipment malfunctions, such as toner replacements.

Safety protocols were adhered to, including proper handling to avoid injuries and maintaining cleanliness to prevent dust buildup. Learning about energy-efficient settings on printers contributed to sustainable practices. Overall, these hands-on experiences provided practical knowledge of office technology, complementing digital skills.

\section{Weekly Work Structure}
Activities followed a general weekly structure, typically from Monday to Friday, with occasional variations based on workload and customer demand. Each workday involved a mix of customer service, document preparation, design tasks, and equipment operation.

Early in the training, emphasis was on observation and guided assistance to understand workplace procedures and tools. As the training progressed, greater responsibility was taken in handling customer requests independently, performing UI/UX design tasks, preparing LaTeX documents, and managing printing operations. This gradual increase in responsibility supported continuous skill development and effective integration into the organization's workflow.

To elaborate, Mondays often started with reviewing pending tasks from the weekend, such as completing overnight print jobs. Mid-week saw peaks in design requests, while Fridays involved wrapping up weekly projects and preparing for the next week. Special events, like university exam periods, increased workload, requiring extended hours and adaptability.

\section{Challenges Faced and Solutions Implemented}
Throughout the internship, various challenges were encountered that tested problem-solving abilities. For example, in UI/UX design, tight deadlines for client deliverables sometimes led to rushed iterations. This was mitigated by prioritizing core features and using Figma's version history for quick rollbacks.

In LaTeX typesetting, compatibility issues with client-provided files were common. Solutions included using Overleaf for cloud-based editing and learning advanced debugging techniques. Customer service challenges, such as handling dissatisfied clients, were addressed through active listening and offering complimentary revisions.

These experiences highlighted the importance of resilience and continuous learning, turning obstacles into opportunities for growth.

\section{Skills Acquired and Professional Development}
The SIWES program facilitated the acquisition of both technical and soft skills. Technical proficiencies improved in Figma, LaTeX, and office equipment operation. Soft skills, including communication, time management, and teamwork, were refined through daily interactions.

Professional development was evident in increased confidence in handling independent projects and understanding workplace ethics. Networking with cafe staff and clients provided insights into industry trends, preparing for future career pursuits in IT and design.

\section{Conclusion and Reflections}
In conclusion, the internship at Litashub Cyber Cafe was a valuable experience that bridged theoretical knowledge with practical application. Reflections reveal personal growth in adaptability and initiative, with recommendations for future interns to actively seek diverse tasks for holistic development.