\chapter{Detailed Intern's Role, Responsibilities and Daily Activities}

\section{Introduction}
This chapter provides a detailed description of the roles, responsibilities, and daily activities carried out during the Students Industrial Work Experience Scheme (SIWES) at Litashub Cyber Cafe. The focus is on the practical tasks performed, the tools used, and the nature of work handled throughout the training period. The presentation is descriptive, highlighting UI/UX design, LaTeX typesetting, customer interaction, and the operation of office equipment.

\section{UI/UX Design Activities}
During the industrial training period, the intern was actively involved in UI/UX design activities using Figma. Prior knowledge of UI/UX principles provided a foundation that was strengthened through continuous practice and exposure to real-world design scenarios.

Using Figma enabled the creation of structured layouts, organized interface elements, and maintained visual consistency across designs. Activities included simple mobile application screens, website layout designs, and visual mock-ups that demonstrated interface structure and user flow. Flyer and poster layouts were also produced for various events and announcements.

Throughout these activities, key UI/UX principles such as alignment, spacing, consistency, typography, and thoughtful color usage were applied. Emphasis was placed on clarity and visual appeal to help customers easily understand and approve design outputs.

\subsection{Projects}

\subsubsection{Exploring the Figma Interface Further}

This was an interactive training session with the supervisor, not a client request. It served as hands-on skill development to strengthen foundational UI/UX capabilities.

The training session focused on exploring advanced features of the Figma interface. Activities included practicing with the pen tool to create complex shapes and curves, designing logos using vector manipulation tools, and creating simple business cards to understand layout principles and spatial composition. The exercises emphasized precision, visual balance, and effective use of design elements.

Key tools and features utilized during this training included the \textbf{Figma pen tool} for vector drawing and creating custom shapes, \textbf{frames} for organizing design elements into structured layouts, \textbf{text objects} for typography practice and understanding font hierarchies, and various \textbf{vector manipulation tools} for editing and refining shapes.

Creating Bézier curves proved difficult initially, requiring practice to master smooth curves and anchor point manipulation. Understanding how control handles affected curve shapes demanded repeated attempts and careful observation. Through continued practice, proficiency improved gradually.

The training was evaluated by the supervisor. Performance was satisfactory though there was room for improvement in mastering complex vector shapes. The session successfully built confidence in using Figma's core design tools and prepared the intern for client-facing design work.

\begin{figure}[h]
\centering
\includegraphics[width=0.8\textwidth]{images/figma_workspace.png}
\caption{Figma workspace showing UI design development and tool exploration}
\label{fig:figma_workspace}
\end{figure}

\subsubsection{Skill Level Display}

This project was an opportunity to demonstrate existing Figma capabilities to the supervisor before handling client work. It served as a practical assessment of design skills and aesthetic judgment.

The project involved creating three burial invitation card templates from scratch. Activities included applying layout design principles to create somber and respectful designs appropriate for memorial contexts, selecting appropriate typography that conveyed dignity and solemnity, and choosing color schemes that reflected the nature of memorial cards. Each template featured distinct layouts while maintaining thematic consistency.

Tools and features used included \textbf{frames} for establishing card structure and organizing content areas, the \textbf{SVG library} for accessing decorative elements and symbols appropriate for memorial cards, \textbf{rich font sets} for selecting elegant typography that suited the formal tone, and the \textbf{lines and pen tool} for creating borders and decorative elements that enhanced visual appeal.

Finding appropriate SVG assets that conveyed the right tone for burial invitations presented a challenge. Many available graphics were either too cheerful or insufficiently respectful for the context. Color selection was also challenging, as memorial cards required subdued, dignified palettes. These challenges were resolved by researching online examples of memorial card designs for inspiration and gathering appropriate visual references.

Successfully created three well-crafted burial invitation templates with carefully chosen layouts and visual elements. The supervisor commended the work, particularly noting the appropriate tone and professional presentation. The templates demonstrated competence in handling sensitive design projects and understanding context-appropriate aesthetics.

\begin{figure}[h]
\centering
\includegraphics[width=0.85\textwidth,height=0.65\textheight,keepaspectratio]{images/burial_invitations.png}
\caption{Sample burial invitation card designs demonstrating layout and typography skills}
\label{fig:burial_invitations}
\end{figure}

\subsubsection{My First Client Request}

This was the first client-facing project assigned by the supervisor—a graduation invitation card requested by a customer. It marked a significant milestone in transitioning from training exercises to real customer work.

Activities performed included meeting with the client to understand specific requirements and preferences, designing a graduation invitation card based on the client's hardcopy sample, incorporating requested visual elements and text content, and making revisions based on client feedback received during review sessions. The project required careful attention to detail and effective communication to ensure client satisfaction.

Tools and features utilized included \textbf{rectangles} for creating card structure and defining layout zones for text and imagery, \textbf{frames} for organizing design elements and maintaining alignment, \textbf{vector objects} for adding decorative elements that enhanced visual appeal, and the \textbf{pen tool} for creating custom shapes and design accents that matched the celebratory theme.

The client provided a hardcopy sample as reference, making color extraction difficult. Matching colors accurately without digital references required multiple attempts and careful visual comparison. The process involved creating color swatches, testing them against the physical sample under different lighting conditions, and making iterative adjustments until acceptable matches were achieved.

The client was highly satisfied with the final design and delivery speed. She provided a tip and made only minor modifications to the approved design. The efficient turnaround time and quality were particularly appreciated. This successful first client interaction built confidence and demonstrated the ability to meet customer expectations under real working conditions.

\begin{figure}[h]
\centering
\includegraphics[width=0.65\textwidth,height=0.55\textheight,keepaspectratio]{images/graduation_invitation.png}
\caption{Graduation invitation card designed for first client request}
\label{fig:graduation_invitation}
\end{figure}

\subsubsection{Litas Travels UI Design}

After training shifted to full UI/UX design focus, the supervisor assigned a project to design a user interface completely from scratch for Litashub's travel services division, Litas Travels. This project represented a significant increase in scope and complexity compared to previous assignments.

The project involved designing multiple key pages for the travel platform. A landing page was created featuring image carousels showcasing travel destinations, call-to-action buttons strategically placed for bookings and inquiries, abstract curves and decorative elements for visual appeal, and catchy, high-quality travel imagery that conveyed excitement and adventure. Additionally, a service description page was designed with proper typesetting for detailed text content, a three-column grid layout for organizing service information, and sticky action items occupying one column to provide persistent navigation and quick access to important functions.

Key tools and features employed included \textbf{auto layouts} for creating responsive design structures that adapted to different content lengths, \textbf{grid markers} for achieving precise alignment and maintaining visual consistency, \textbf{breakpoints} for ensuring responsive behavior across different device sizes, \textbf{ellipses and vector objects} for adding decorative elements and visual interest, and \textbf{components} for creating reusable UI elements that ensured consistency throughout the design.

Designing a complete interface from scratch without a reference was challenging. It required understanding of modern web design patterns, user expectations for travel websites, and effective information architecture. Research was conducted to gather design inspiration from modern travel websites and contemporary UI design patterns. Analysis of competitor websites and design showcases informed layout decisions and feature prioritization.

Successfully delivered a modern, visually appealing landing page and service page that met the organization's needs for promoting travel services. The design effectively communicated the brand's value proposition and provided clear pathways for user engagement. The project demonstrated the ability to create comprehensive UI designs from initial concept to final deliverable.

\begin{figure}[h]
\centering
\includegraphics[height=0.8\textheight,keepaspectratio]{images/litas_travels_landing_page.png}
\caption{Landing page design for Litas Travels service platform}
\label{fig:litas_travels_landing_page}
\end{figure}

\subsubsection{Final Year Project Interface Design}

As a culminating project, the intern was allowed to work on a project of personal choice. The selected project was designing the user interface for an intended final year project: an attendance recording system. This project provided an opportunity to apply accumulated skills to a personally meaningful application.

The project involved designing nine key screens highlighting the registrar user flow. The mobile interface featured bottom tab navigation for intuitive screen switching. Specific screens and components designed included sign-up and sign-in screens with appropriate form validation indicators, a dashboard displaying attendance cards with summary statistics, headers with navigation elements and contextual actions, dropdowns and popovers for making selections and filtering data, dialog boxes for confirmations and critical user decisions, and helper texts with color markers for destructive actions to prevent accidental data loss.

Tools and features utilized included \textbf{auto layouts} for creating responsive mobile design structures that adapted gracefully, \textbf{frames and sections} for organizing screens and maintaining clear visual hierarchy, \textbf{components} for creating reusable UI elements such as buttons, cards, and navigation items that ensured consistency, \textbf{absolute positioning} for overlay elements like modals and tooltips, \textbf{shadows} for adding depth and establishing visual hierarchy, and the \textbf{Lucide Icon library} for consistent iconography throughout the application.

No significant challenges were faced because a document flow and structural plan had been prepared beforehand, streamlining the design process. The advance planning included user flow diagrams, screen wireframes, and component inventories. This preparation enabled efficient execution and reduced design iterations.

Successfully created a comprehensive nine-screen mobile application prototype demonstrating complete user flow for an attendance recording system. The design was well-structured, visually cohesive, and ready for development handoff. The project showcased the ability to design complete application interfaces with complex interactions and multiple user scenarios.

\begin{figure}[h]
\centering
\includegraphics[width=0.85\textwidth,height=0.75\textheight,keepaspectratio]{images/attendance_system_highlight.png}
\caption{Mobile application interface screens for attendance recording system}
\label{fig:attendance_system_highlight}
\end{figure}

\begin{figure}[h]
\centering
\includegraphics[width=0.8\textwidth,height=0.6\textheight,keepaspectratio]{images/reusable_components.png}
\caption{Reusable UI components library created in Figma}
\label{fig:reusable_components}
\end{figure}

\section{LaTeX Typesetting Activities}
LaTeX typesetting constituted a significant part of the responsibilities at Litashub Cyber Cafe. The work involved preparing and formatting academic and professional documents using LaTeX, a system widely used for producing high-quality structured documents.

Documents handled included student assignments, examination materials, project reports, and academic notes. Both editing existing LaTeX files and creating new documents from scratch were performed.

Common LaTeX features used included sectioning and subsectioning, page numbering, table of contents generation, referencing, insertion of tables and figures, and formatting of mathematical equations. Additional packages such as amsmath, hyperref, and biblatex were frequently employed to enhance document functionality.

Litashub adopted LaTeX for academic documents because it produced clean, professional, and standardized outputs. The system was especially useful for project reports submitted by university students, ensuring consistency with institutional requirements.

\subsection{Projects}
% TODO: Add specific LaTeX typesetting project descriptions

\section{Customer Interaction and Service}
Customer interaction formed a major aspect of daily activities at Litashub Cyber Cafe. The intern interacted directly with customers, receiving requests, clarifying requirements, and providing assistance related to typing, printing, design, and document preparation services.

Typical requests included document typing, printing, corrections, redesigns, and minor formatting adjustments. Some customers had clear specifications, while others needed guidance on suitable layouts or presentation styles. In such cases, options were communicated and design or formatting choices explained to ensure satisfaction.

Customers frequently requested changes after reviewing completed outputs, either due to revised preferences or initial oversight. This required attentiveness, patience, and flexibility while handling multiple tasks simultaneously.

\subsection{Typical Customer Requests}

\subsubsection{Urgent Typing Assignment - Personal Speed Milestone}

A customer arrived at Litashub with urgent typing work that needed to be completed quickly. The document was handwritten and required accurate transcription within a tight timeframe.

The customer presented several pages of handwritten notes that needed to be typed, and the document had to be completed the same day for submission. The intern had available time and accepted the assignment.

Activities included assessing the handwritten document to understand structure and content, then setting up the document in Microsoft Word with appropriate formatting. Typing began systematically while maintaining focus and accuracy. During this session, a personal typing speed milestone of \textbf{90 words per minute (WPM)} was achieved. The completed document was proofread for accuracy before delivery, and the typed document was delivered within the promised timeframe.

Balancing speed with accuracy to ensure no transcription errors while working under time pressure presented a challenge. Maintaining concentration throughout the extended typing session required discipline and focus.

Successfully completing the urgent typing assignment within the deadline resulted in customer satisfaction with both the accuracy and quick turnaround time. This experience demonstrated improved typing proficiency and the ability to handle time-sensitive tasks efficiently. Achieving 90 WPM marked a significant personal milestone in developing professional typing skills and proved the value of consistent practice.

\subsubsection{Legal Document Revisions - Precision Under Scrutiny}

The legal document that had been typed for court submission (mentioned in Section 2.2.2) required multiple revisions based on client review and legal requirements. Unlike design work where revisions involve aesthetic changes, legal document revisions demanded absolute precision.

After initial typing of the legal document, the client returned with specific corrections. Revisions included exact wording changes in legal clauses, correction of names, dates, and case numbers, adjustment of numbered clause sequences, and modification of formal legal language. Each revision had to be implemented with zero tolerance for error.

Activities performed included carefully reviewing client's revision notes and markups, then making precise changes to specific sections without altering surrounding content. Every modification was cross-checked against client's instructions. Clause numbering was verified to remain consistent after insertions or deletions. Draft copies were printed for client review before final submission. Additional corrections were made based on subsequent reviews, and detailed tracking of changes across revision cycles was maintained.

Maintaining absolute accuracy across multiple revision cycles proved demanding. Legal documents cannot contain any typing errors, misplaced punctuation, or incorrect formatting, as these could have serious legal implications. Every change had to be tracked carefully to ensure nothing was missed or incorrectly modified. The pressure of knowing that errors could affect legal proceedings required exceptional attention to detail.

Successfully completing all required revisions with zero errors in the final document meant the legal document was accepted by the court without issues. This experience emphasized the critical importance of precision, attention to detail, and careful proofreading in professional document preparation. It demonstrated the ability to work under scrutiny where mistakes could have significant consequences beyond customer satisfaction.

\section{Operation of Office Machines}
Beyond design and typesetting tasks, the intern operated office machines used at Litashub Cyber Cafe. Standard HP LaserJet printers and photocopy machines supported daily operations, ensuring smooth workflow.

Routine tasks included printing documents, reloading paper trays, adjusting printer settings for different paper sizes and orientations, and clearing minor paper jams. Photocopying was performed as required by customers. These activities built familiarity with basic office equipment and routine maintenance procedures.

\subsection{Printing/Binding Examples}

The following examples demonstrate equipment operation skills and the ability to manage high-volume production tasks while maintaining quality standards.

\subsubsection{Batch Printing of Student Clearance Slips}

During clearance periods, dozens of students required printed clearance slips for various departments and administrative processes. This created high-volume printing demand that needed efficient handling.

Activities included receiving the digital clearance slip template from university administrative contacts. Students arrived individually or in small groups requesting clearance prints. Batch printing requests were processed throughout the day. Clearance slips were printed in rapid succession during peak periods. Each slip required student-specific information (name, matric number, department). Printed slips were organized systematically to avoid mix-ups among multiple customers.

Tools and equipment used included:
\begin{itemize}
\item HP LaserJet printer for quick document printing
\item Microsoft Word or PDF files containing clearance templates
\item Paper trays stocked with standard A4 paper
\end{itemize}

Several challenges arose during this process. Managing high volume of students during peak clearance periods required efficient workflow. Ensuring each student received the correct personalized clearance slip was critical. Maintaining printer operation continuously without overheating during sustained use demanded careful monitoring. Handling impatient students waiting for urgent clearance documents required patience and good customer service. Paper and toner management during sustained high-volume printing sessions needed attention, and queue management was necessary to maintain order during busy periods.

Successfully processing clearance printing for dozens of students in single sessions demonstrated the ability to handle high-volume tasks. Efficient workflow for handling repetitive printing tasks was developed. Experience was gained in managing customer queues, maintaining equipment under heavy use, and organizing output to prevent errors. The experience demonstrated the ability to work efficiently under pressure while maintaining accuracy in document handling.

\subsubsection{Certificate Printing and Lamination}

A church organization requested printing and lamination of 50 certificates for participants in a completed training program.

Activities performed included receiving the certificate template designed in Microsoft Word with recipient names in an Excel file. Mail merge was performed to generate 50 personalized certificates. Certificates were printed on special certificate paper (A4 size, high-quality cardstock). Printer settings were adjusted for thick paper to prevent jamming. Each printed certificate was inspected for quality (alignment, color accuracy, text clarity). Certificates were trimmed to standard size using a paper cutter. All 50 certificates were laminated using a laminating machine with appropriate pouch size. Lamination quality was ensured to be smooth without air bubbles or wrinkles.

Tools and equipment used included:
\begin{itemize}
\item Printer with manual feed capability for thick certificate paper
\item Laminating machine with heat control settings
\item Paper cutter for precise trimming
\end{itemize}

Several challenges were encountered. Certificate paper was thicker than standard paper, requiring printer setting adjustments to prevent damage. Initial test print revealed alignment issues requiring template modification. Laminating machine temperature needed calibration to prevent overheating that could warp certificates. Ensuring each certificate was perfectly centered during lamination process required careful handling. Time management to complete 50 certificates within agreed timeframe was demanding. Maintaining quality consistency across all 50 certificates required vigilance.

Delivered 50 professionally printed and laminated certificates. The organization was highly satisfied with the quality and professional appearance. Certificates enhanced the prestige of the training program. Gained experience with specialized printing materials, lamination techniques, quality control for formal documents, and managing multi-step production processes.

\section{Weekly Work Structure}
Activities followed a general weekly structure, typically from Monday to Friday, with occasional variations based on workload and customer demand. Each workday involved a mix of customer service, document preparation, design tasks, and equipment operation.

Early in the training, emphasis was on observation and guided assistance to understand workplace procedures and tools. As the training progressed, greater responsibility was taken in handling customer requests independently, performing UI/UX design tasks, preparing LaTeX documents, and managing printing operations.