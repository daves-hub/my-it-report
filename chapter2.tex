\chapter{Detailed Intern's Role, Responsibilities and Daily Activities}

\section{Introduction}
This chapter provides a detailed description of the roles, responsibilities, and daily activities carried out during the Students Industrial Work Experience Scheme (SIWES) at Litashub Cyber Cafe. The focus is on the practical tasks performed, the tools used, and the nature of work handled throughout the training period. The presentation is descriptive, highlighting UI/UX design, LaTeX typesetting, customer interaction, and the operation of office equipment.

\section{UI/UX Design Activities}
During the industrial training period, the intern was actively involved in UI/UX design activities using Figma. Prior knowledge of UI/UX principles provided a foundation that was strengthened through continuous practice and exposure to real-world design scenarios.

Using Figma enabled the creation of structured layouts, organized interface elements, and maintained visual consistency across designs. Activities included simple mobile application screens, website layout designs, and visual mock-ups that demonstrated interface structure and user flow. Flyer and poster layouts were also produced for various events and announcements.

Throughout these activities, key UI/UX principles such as alignment, spacing, consistency, typography, and thoughtful color usage were applied. Emphasis was placed on clarity and visual appeal to help customers easily understand and approve design outputs.

\subsection{Projects}
% TODO: Add specific UI/UX project descriptions

\section{LaTeX Typesetting Activities}
LaTeX typesetting constituted a significant part of the responsibilities at Litashub Cyber Cafe. The work involved preparing and formatting academic and professional documents using LaTeX, a system widely used for producing high-quality structured documents.

Documents handled included student assignments, examination materials, project reports, and academic notes. Both editing existing LaTeX files and creating new documents from scratch were performed.

Common LaTeX features used included sectioning and subsectioning, page numbering, table of contents generation, referencing, insertion of tables and figures, and formatting of mathematical equations. Additional packages such as amsmath, hyperref, and biblatex were frequently employed to enhance document functionality.

Litashub adopted LaTeX for academic documents because it produced clean, professional, and standardized outputs. The system was especially useful for project reports submitted by university students, ensuring consistency with institutional requirements.

\subsection{Projects}
% TODO: Add specific LaTeX typesetting project descriptions

\section{Customer Interaction and Service}
Customer interaction formed a major aspect of daily activities at Litashub Cyber Cafe. The intern interacted directly with customers, receiving requests, clarifying requirements, and providing assistance related to typing, printing, design, and document preparation services.

Typical requests included document typing, printing, corrections, redesigns, and minor formatting adjustments. Some customers had clear specifications, while others needed guidance on suitable layouts or presentation styles. In such cases, options were communicated and design or formatting choices explained to ensure satisfaction.

Customers frequently requested changes after reviewing completed outputs, either due to revised preferences or initial oversight. This required attentiveness, patience, and flexibility while handling multiple tasks simultaneously.

\subsection{Typical Customer Requests}
% TODO: Add detailed examples of customer requests and interactions

\section{Operation of Office Machines}
Beyond design and typesetting tasks, the intern operated office machines used at Litashub Cyber Cafe. Standard HP LaserJet printers and photocopy machines supported daily operations, ensuring smooth workflow.

Routine tasks included printing documents, reloading paper trays, adjusting printer settings for different paper sizes and orientations, and clearing minor paper jams. Photocopying was performed as required by customers. These activities built familiarity with basic office equipment and routine maintenance procedures.

\section{Weekly Work Structure}
Activities followed a general weekly structure, typically from Monday to Friday, with occasional variations based on workload and customer demand. Each workday involved a mix of customer service, document preparation, design tasks, and equipment operation.

Early in the training, emphasis was on observation and guided assistance to understand workplace procedures and tools. As the training progressed, greater responsibility was taken in handling customer requests independently, performing UI/UX design tasks, preparing LaTeX documents, and managing printing operations.

% Project sections reserved for specific project descriptions
\section{Project Six}
% TODO: Add project description

\section{Project Seven}
% TODO: Add project description

\section{Project Eight}
% TODO: Add project description

\section{Project Nine}
% TODO: Add project description

\section{Project Ten}
% TODO: Add project description

\section{Project Eleven}
% TODO: Add project description

\section{Project Twelve}
% TODO: Add project description

\section{Project Thirteen}
% TODO: Add project description

\section{Project Fourteen}
% TODO: Add project description

\section{Project Fifteen}
% TODO: Add project description

\section{Project Sixteen}
% TODO: Add project description

\section{Project Seventeen}
% TODO: Add project description

\section{Project Eighteen}
% TODO: Add project description