\chapter{Detailed Intern's Role, Responsibilities and Daily Activities}

\section{Introduction}
This chapter describes the roles and daily activities during the SIWES training at Litashub Cyber Cafe. It focuses on the practical tasks in UI/UX design, LaTeX typesetting, customer service, and office machine operation.

\section{UI/UX Design Activities}
The intern worked on UI/UX design using Figma, where prior knowledge was improved through both practice and real-world projects. Using Figma allowed for the creation of structured layouts and consistent interfaces across various tasks, including mobile app screens, website layouts, and event flyers. Key design principles like alignment, spacing, and color theory were consistently applied to create clear, attractive, and user-friendly designs for customers.

\subsection{Projects}

\subsubsection{Exploring the Figma Interface}
This training session with the supervisor was designed to improve basic skills and introduce advanced Figma features. Exercises focused on practical applications, including using the pen tool for shapes, designing logos, and creating business cards. Key tools utilized during this session included the pen tool for vector drawing, frames for layout structure, text objects for typography, and vector tools for shape editing. While creating smooth curves with the pen tool was initially challenging, consistent practice led to significant improvement. The supervisor evaluated the work as satisfactory, noting that although complex shapes needed more practice, the session successfully built the confidence required for future client work.

\begin{figure}[h]
\centering
\includegraphics[width=0.8\textwidth]{images/figma_workspace.png}
\caption{Figma workspace showing UI design development and tool exploration}
\label{fig:figma_workspace}
\end{figure}

\subsubsection{Skill Display Project}
This project served as a practical test given by the supervisor before starting client work. The task required creating three burial invitation cards from scratch, with a focus on respectful layouts, appropriate fonts, and somber colors. Tools used included frames for structure, the SVG library for symbols, and the pen tool for borders. Finding suitable SVG graphics proved challenging, as many readily available options were too cheerful for the context. However, researching existing designs helped in finding the right style. The supervisor praised the final designs for their professional tone, confirming readiness for sensitive client projects.

\begin{figure}[h]
\centering
\includegraphics[width=0.85\textwidth,height=0.65\textheight,keepaspectratio]{images/burial_invitations.png}
\caption{Sample burial invitation card designs demonstrating layout and typography skills}
\label{fig:burial_invitations}
\end{figure}

\subsubsection{First Client Request}
The first real client project involved designing a graduation invitation card. The task required discussing requirements directly with the client, replicating a hardcopy sample design, and making revisions based on feedback. Tools used included rectangles for layout, frames for alignment, and the pen tool for creating custom shapes. Matching colors from the physical sample was difficult and required a trial-and-error approach. Ultimately, the client was very happy with the result and the speed of delivery, marking a smooth transition to professional work.

\subsubsection{Litas Travels UI Design}
The supervisor assigned a major project: designing the interface for Litas Travels from scratch. The work included designing a landing page with image carousels and a service page with a grid layout, aiming for a modern, travel-themed interface. Key tools employed were auto layouts for responsiveness, grid markers for alignment, and components for reusable elements. Designing without a reference was challenging, so researching other travel websites helped guide the layout and style decisions. The final design was modern and effective, successfully meeting the business needs.
\begin{figure}[htbp]
\centering
\includegraphics[width=1.0\textwidth]{images/litas_travels_landing_page.png}
\caption{Litas Travels landing page design showcasing modern UI elements}
\label{fig:litas_travels}
\end{figure}
\subsubsection{Final Year Project Interface}
The intern chose to design the interface for a final year project: an attendance recording system. Nine screens were designed for the registrar flow, including sign-up, dashboard, and data filtering screens. The project relied heavily on auto layouts, components for consistency, and the Lucide Icon library. Advance planning with wireframes ensured a smooth execution with few challenges, resulting in a complete, nine-screen mobile app prototype ready for development.


\begin{figure}[h]
\centering
\includegraphics[width=0.8\textwidth,height=0.6\textheight,keepaspectratio]{images/reusable_components.png}
\caption{Reusable UI components library created in Figma}
\label{fig:reusable_components}
\end{figure}

\section{LaTeX Typesetting Activities}
LaTeX was used extensively for accurate document preparation, including student assignments, project reports, and exams. Common features used included sectioning, tables of contents, references, tables, and equations, with packages like \texttt{amsmath} and \texttt{geometry} being standard. LaTeX was consistently preferred over other tools for its clean, professional output, especially for academic work.

\begin{figure}[h]
\centering
\includegraphics[width=0.85\textwidth]{images/latex_editor.png}
\caption{LaTeX editor workspace showing document preparation}
\label{fig:latex_editor}
\end{figure}

\subsection{Projects}

\subsubsection{Ph.D Assignment}
A Ph.D student needed a multi-page assignment typed from handwritten notes. The task involved creating footnotes and a bibliography. Learning the humanities citation style, which uses extensive footnotes, was new and required quick research. The student was satisfied with the properly formatted document.

\subsubsection{Legal Document}
A client needed a legal document for court submission, which required high accuracy. The work involved typing clauses, ensuring correct numbering with the \texttt{enumitem} package, and adhering to strict formatting guidelines. Precision was critical; every word and comma had to be correct. The document was accepted by the court, and the client was satisfied with the professional result.

\subsubsection{Math Exam Paper}
A teacher requested a math question paper to be typed. This involved formatting equations, fractions, and symbols using the \texttt{amsmath} package. Determining the correct LaTeX syntax for complex math symbols required quick learning, but the teach was impressed by the clear layout and accurate notation.

\begin{figure}[h]
\centering
\includegraphics[width=0.8\textwidth]{images/latex_mathematics.png}
\caption{Mathematical equations formatted in LaTeX}
\label{fig:latex_mathematics}
\end{figure}

\subsubsection{Personal Math Practice}
This self-study session was undertaken to improve math typesetting skills. Practice included complex equations, matrices, integrals, and limits using \texttt{amsmath} and \texttt{amssymb}. Mastering the syntax required dedication, but it built the necessary confidence for future math-heavy projects.

\subsubsection{Student Project Report}
A university student needed their project report formatted to school standards. The work involved creating a title page, automatic table of contents, chapters, and proper page numbering. Matching the specific school guidelines for margins and spacing required careful attention to detail. The report passed the school's review on the first try, leading to more work from the same student.

\begin{figure}[h]
\centering
\includegraphics[width=0.8\textwidth]{images/latex_output_sample.png}
\caption{Sample formatted academic document output}
\label{fig:latex_output_sample}
\end{figure}

\subsubsection{Church Program}
A church needed a program for an event. The design utilized the \texttt{multicol} package for a multi-column layout to optimize space. Fitting a large amount of information onto a few pages while maintaining readability was the main challenge, but the final program was clean and easy to read.

\begin{table}[h]
\centering
\caption{Common LaTeX packages utilized during typesetting activities}
\begin{tabular}{|l|p{6cm}|l|}
  \hline
\textbf{Package} & \textbf{Purpose} & \textbf{Frequency} \\
\hline
amsmath & Mathematical equations and expressions & High \\
\hline
amssymb & Additional mathematical symbols & Medium \\
\hline
graphicx & Including images and figures & Medium \\
\hline
hyperref & Creating hyperlinks in PDFs & High \\
\hline
geometry & Page layout and margin control & High \\
\hline
multicol & Multi-column layouts & Low \\
\hline
enumitem & Enhanced list formatting and customization & Medium \\
\hline
biblatex & Bibliography management & Medium \\
\hline
\end{tabular}
\label{tab:latex_packages}
\end{table}

\section{Customer Interaction}
The intern assisted customers with typing, printing, and design requests daily. These requests ranged from simple typing jobs to design changes. Clear communication was essential to understand requirements, especially when they were vague. Handling revisions and customer feedback required patience and attention to detail.

\subsection{Customer Request Examples}

\subsubsection{Urgent Typing - 90 WPM Milestone}
A customer needed handwritten notes typed urgently. The typing was done quickly in Microsoft Word, during which a personal speed of \textbf{90 WPM} was achieved. Maintaining accuracy at such high speeds was key, and the work was finished on time without errors.

\subsubsection{Legal Document Revisions}
The legal document mentioned earlier required several rounds of corrections. These revisions were critical, as names, dates, and numbers had to be exact. Accuracy was the priority, and detailed checking ensured no errors were made in the final version.

\section{Operation of Office Machines}
Daily tasks included operating HP LaserJet printers and photocopiers. Common duties involved printing documents, loading paper, and clearing jams, providing basic equipment maintenance experience.

\subsection{Printing Examples}

\subsubsection{Batch Printing Clearance Slips}
During clearance periods, many students needed slips printed. This required efficient batch printing from templates while ensuring the right slip was matched to the right student. Challenges included printer overheating and managing the student queue, but an efficient workflow helped handle the volume successfully.

\subsubsection{Certificate Printing and Lamination}
A church needed 50 certificates printed and laminated. The work involved using mail merge for names, printing on cardstock, and laminating each certificate. Settings had to be adjusted for the thick paper, and lamination required care to avoid bubbles. All 50 certificates were delivered successfully, to the client's satisfaction.

\section{Weekly Work Structure}
Work typically ran from Monday to Friday. Days were filled with customer service, design, printing, and typing tasks. Responsibilities grew from observation to independent work as the internship progressed.