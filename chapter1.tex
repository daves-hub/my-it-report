\chapter{Introduction}

\section{Overview of the Students Industrial Work Experience Scheme (SIWES)}

The Students Industrial Work Experience Scheme (SIWES) is a practical training programme designed to expose undergraduate students to the real working environment related to their course of study. It serves as an integral component of the Nigerian tertiary education system, aimed at bridging the gap between theoretical knowledge acquired in the classroom and practical application in industry.

SIWES provides students with the opportunity to acquire hands-on experience, develop technical competence, and understand professional work ethics before graduation. Through active participation in industrial activities, students are better prepared for future careers and are equipped with relevant skills required by employers in their respective fields.

The scheme also enables students to interact with professionals, adapt to workplace culture, and develop problem-solving abilities in real-life scenarios. Overall, SIWES plays a vital role in enhancing students’ employability and professional development.

\section{History and Background of the Organization}

Litashub Cyber Café is a private digital service enterprise established to provide essential information technology and documentation services to students, businesses, and members of the general public. The organization operates primarily as a cyber café, offering services such as document preparation, printing, photocopying, internet access, and basic digital design solutions.

Over time, Litashub has expanded its scope of services to include academic typesetting, graphical layout design, and user interface planning for simple digital projects. The organization serves a diverse range of customers, including university students, lecturers, entrepreneurs, and individuals requiring professional document and design services.

Litashub operates in a fast-paced service environment and utilizes both hardware and software tools to deliver quality output efficiently. The organization maintains a modest workforce consisting of a supervisor and operational staff responsible for customer service, design, and equipment handling.

\section{Organizational Structure and Responsibilities}

The organizational structure of Litashub Cyber Café is relatively simple and hierarchical, designed to ensure efficient service delivery and accountability. At the top of the structure is the supervisor, who oversees daily operations, manages staff activities, and ensures customer satisfaction.

Operational staff members are responsible for attending to customers, carrying out document preparation, operating office equipment such as printers and photocopy machines, and handling basic design and formatting tasks. Each staff member plays a role in ensuring that customer requests are executed accurately and promptly.

During the industrial training period, the intern functioned under the supervision of the organization’s supervisor and worked closely with operational staff, integrating into the existing workflow of the organization.

\section{Summary of the Intern’s Role and Responsibilities}

During the Students Industrial Work Experience Scheme at Litashub Cyber Café, the intern was actively involved in various technical and service-oriented tasks, with particular emphasis on digital design and document preparation.

The primary responsibilities of the intern included assisting in user interface and layout design using Figma, preparing academic and professional documents through LaTeX typesetting, and ensuring that documents adhered to standard formatting requirements. The intern also interacted with customers to understand their design and documentation needs, translating these requirements into well-structured digital outputs.

In addition, the intern participated in routine operational activities such as operating printers and photocopy machines, managing document output, and maintaining a professional working relationship with customers and staff. These responsibilities provided valuable exposure to both technical skill development and workplace professionalism.

Overall, the internship enabled the intern to apply theoretical knowledge in a practical setting while developing competence in UI/UX design, LaTeX typesetting, and effective customer interaction.
