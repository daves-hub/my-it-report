\chapter{Introduction}

\section{Brief Introduction about SIWES, Aims and Objectives}
The \textbf{Student Industrial Work Experience Scheme (SIWES)} is a program that forms part of the academic standards in the degree program for Nigerian Universities. The \textbf{Federal Government of Nigeria} introduced the policy on Industrial training, called the Student Industrial Work Experience Scheme (SIWES) in \textbf{1974}. The \textbf{Industrial Training Fund (ITF)} is in charge of this program which is under the umbrella of the \textbf{Ministry of Education}. SIWES was designed to help students acquire the necessary \textbf{practical education/experience} in their fields of study and other related professions.

This is an effort that was created in order to \textbf{complement the theory} taught in the classrooms of the Nigerian tertiary institutions. The objective of the program is to expose students to the use of various machines and equipment, professional work methods, and ways of safeguarding the work areas in industries as well as other organizations. SIWES was established to impart practical knowledge to students with respect to their various disciplines.

This training is funded by the Federal Government of Nigeria and coordinated by the Industrial Training Fund (ITF) and the \textbf{National Universities Commission (NUC)}. The SIWES program involves the student, the universities, and the industries.

\subsection{Objectives of SIWES}
\begin{itemize}
    \item The program teaches the student how to interact effectively with other workers and supervisors under various conditions in the organization.
    \item It will help students to gain increased maturity and understanding of the workplace.
    \item The students will have a chance to evaluate companies for which they might wish to work.
    \item It exposes students to work methods and techniques in handling equipment and machines that may not be available in the educational institution.
    \item The program provides students with an opportunity to apply their knowledge in real work and actual practice.
    \item SIWES increases a student’s sense of responsibility.
    \item SIWES provides students the opportunity to test their interest in a particular career before permanent commitments are made.
    \item It helps them to gain interpersonal and entrepreneurial skills.
\end{itemize}

\section{History and Background of the Organization}
Litashub Cyber Café, also known as \textbf{LITAS Hub}, is a private digital service business established in \textbf{2019} in \textbf{Benin City, Edo State, Nigeria}. Founded by \textbf{Eng. Ogbebor Osakpolor Survival}, its mission is to provide accessible and efficient IT services to students, small businesses, and the public. Initially starting as a cyber café to meet the demand for internet access and computer services, its core offerings included browsing, typing, printing, photocopying, scanning, and lamination.

Over time, Litashub has expanded significantly to meet evolving customer needs. beyond basic cyber café services, it now offers assistance with NYSC registration, online payments, and basic design, while also diversifying into logistics, travel services, farming, laundry, and cryptocurrency exchange. Serving a diverse clientele of students, lecturers, entrepreneurs, and the general public, Litashub utilizes modern computers and equipment to deliver quality work. The organization operates with a simple structure, led by a supervisor who manages daily operations and staff who handle customer service, document processing, and design, solidifying its reputation as a reliable digital solutions provider in the area.

\section{Summary of the Intern’s Role and Responsibilities}
During the SIWES program at Litashub Cyber Café, the intern worked on various technical tasks, focusing mainly on digital design and document preparation. Primary duties included UI/UX design using Figma, preparing documents with LaTeX, and ensuring standard formatting. This involved working closely with customers to understand their unique needs and create suitable designs. Additionally, the intern handled daily operational tasks like operating printers and photocopiers, managing documents, and maintaining good relationships with both customers and staff. This combination of technical responsibilities and client interaction helped develop a well-rounded skillset, allowing the intern to apply classroom knowledge in a real setting while improving proficiency in design, LaTeX, and customer service.
